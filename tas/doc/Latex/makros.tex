%========================================================================================================
% Eigene Befehle

% Mathematische Schreibweisen
% Vektoren: fettgedruckt (auskommentieren, um Vektoren mit Pfeil zu erhalten)
\renewcommand{\vec}[1]{\boldsymbol{#1}}
% Matrizen: auch fettgedruckt
\newcommand{\mat}[1]{\boldsymbol{#1}}
% partielle Ableitung
\newcommand{\partder}[2]{\frac{\partial #1}{\partial #2}} 
% Br�che mit schr�gem Bruchstrich
\newcommand{\nfrac}[2]{
  \leavevmode\kern.1em%
  \raise.5ex\hbox{\scriptsize #1}%
  \kern-.1em/\kern-.15em%
  \lower.25ex\hbox{\scriptsize #2}
}

% Todo-Anmerkungen machen und auflisten
\newcommand{\todolistname}{To Do:}
\newlistof{todo}{todo}{\todolistname}
\newcommand{\todo}[1] {%
  \refstepcounter{todo}
  \textbf{\color{red} TODO: #1!}
  \addcontentsline{todo}{todo}{#1}\par}

% Verweise auf Gleichungen, Abbildungen, Kapitel
% mit ref
\newcommand{\refeq}[1]{~\ref{eq:#1}}
\newcommand{\reffig}[1]{~\ref{fig:#1}}
\newcommand{\refsec}[1]{~\ref{sec:#1}}
\newcommand{\reftab}[1]{~\ref{tab:#1}}
% und vref
\newcommand{\vrefeq}[1]{~\vref{eq:#1}}
\newcommand{\vreffig}[1]{~\vref{fig:#1}}
\newcommand{\vrefsec}[1]{~\vref{sec:#1}}
\newcommand{\vreftab}[1]{~\vref{tab:#1}}

% nichttrennender Bindestrich
\newcommand{\ntb}{\nobreakdash-\hspace{0pt}}
